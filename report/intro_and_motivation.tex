% \usepackage{comment}


Disastrous events spread in networks and might lead to total breakdown of the system,  for example, road infrastructures, power system and social networks. Emergency response and recovery call for external resources are crucial for survival and determine whether the affected areas will overcome the consequences of catastrophes or not \cite{buzna2007efficient}. The usual situation is that we have limited resources and we need to make a decision on the resource distribution. A good strategy is very important for overcoming a disaster and may prevent us from enormous economic loss. 

In this work, we study the problem of effectively utilizing limited external resources to control disaster spreading in a network. In the previous work of Buzan, et. al \cite{buzna2007efficient}, they compare six heuristic strategies to distribute resources when disaster happens and analyze the ``best" one among these six strategies for different scenarios. However, there exist lots of other heuristic strategies since the ways to distribute resources are infinite. So a natural question is: what is the optimal strategy? To our knowledge, no previous work has answered this question.

We find this problem could be modeled as a PDE-constrained optimization problem given the network topology. We proposed two different objective functions for different optimal criteria. By using adjoint method to solve this optimization problem, we finally obtain the optimal way to distribute resources when disaster happens. Numerical results show that the optimal strategy is much more effective than these heuristic strategies studied in \cite{buzna2007efficient}.

This report is organized as the following way: We firstly introduce the disaster spreading model in Sec. \ref{sec:spreadingmodel}. In Sec. \ref{sec:methods}, we discuss the heuristic strategies proposed in \cite{buzna2007efficient} and formulate our optimization problem. After this, in Sec. \ref{sec:adjointmethod} we introduce the adjoint method in a slightly general way to avoid complex settings in our model. Sec. \ref{sec:implementation} is our experiments design, followed by results in Sec. \ref{sec:results}. We propose some interesting work that can be done in further studies in Sec. \ref{sec:summary}. We present the adjoint system of our model in Appendix \RN{1} and explain our code structure in Appendix \RN{2}.
